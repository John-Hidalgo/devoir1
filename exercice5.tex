\documentclass{article}
\usepackage{amsmath}
\usepackage{amssymb}
\usepackage[margin=1in]{geometry}
\usepackage{array} 

\begin{document}

Exercice 5

\vspace{0.5cm}
Pour cet exercice, nous utiliserons les 2 faits suivants sur les inégalités..

$$E_1:(\forall a,b,c \in \mathbb{R} \mid a \leq b \iff a - c \leq b - c)$$
$$E_2:(\forall a,b\in \mathbb{R}   \mid a \leq 0 \land b \leq 0 \land a \leq b \implies a^2 \leq b^2)$$

\textbf{Démonstration} Nous voulons montrer que $(\forall x \in \mathbb{Z} \setminus \mathbb{N} \mid (x - 4)^2 \geq 8)$. Soit $ x \in \mathbb{Z} \setminus \mathbb{N}.$ Par définition de la différence d'ensembles, nous avons
$$x \in \mathbb{Z} \land x \notin \mathbb{N} \iff x \in \mathbb{Z} \land x \leq -1 $$
d'après les propriétés des entiers. Maintenant, en appliquant la règle d'équivalence à $E_1[r:= x] \iff E_1[r:= -1]$, nous avons

$$ x \leq -1 \iff x - 4 \leq -1 - 4 $$

En appliquant la règle d'équivalence aux propriétés de l'arithmétique, nous obtenons
$$ x - 4 \leq -1 - 4 \iff x - 4 \leq -5 $$

Puisque $x - 4$ et $-5$ sont tous deux négatifs, nous pouvons appliquer $E_2$ et la règle de substitution avec $E_2[a:=x-4]$ et $E_2[b:=-5]$, ce qui nous donne,

$$ x - 4 \leq -5 \iff (x - 4)^2 \geq 25 $$

Et comme $25 \geq 8$, cela implique immédiatement ce qui était requis.

$$ (x - 4)^2 \geq 8 $$
$\hfill \square$


\end{document