\documentclass{article}
\usepackage{amsmath}
\usepackage[margin=1in]{geometry}
\usepackage{array} 

\begin{document}

Exercice 1

\vspace{0.5cm}
Il est possible de le prouver en effectuant la table de vérité de l'expression:

\begin{center}
\begin{tabular}{| c | c | c | c | c | c | c | c |}
\hline
$p$ & $q$ & $r$ & $p \land q$ & $(p \land q) \implies r$ & $p \implies r$ & $q \implies r$ & $(p \implies r) \land (q \implies r)$ \\
\hline
V & V & V & V & V & V & V & V\\
V & V & F & V & F & F & F & F\\
V & F & V & F & V & V & V & V\\
V & F & F & F & V & F & V & F\\
F & V & V & F & V & V & V & V\\
F & V & F & F & V & V & F & F\\
F & F & V & F & V & V & V & V\\
F & F & F & F & V & V & V & V\\
\hline
\end{tabular}
\end{center}

\vspace{0.5cm}

Pour clarifier, les expressions $(p \land q) \implies r$ et $(p \implies r) \land (q \implies r)$ ne sont pas logiquement équivalentes car elles diffèrent dans les lignes 2 et 4 du tableau de vérité. Plus précisément :

\begin{itemize}
    \item Dans la ligne 2, $(p \land q) \implies r$ est **F** tandis que $(p \implies r) \land (q \implies r)$ est **F**.
    \item Dans la ligne 4, $(p \land q) \implies r$ est **V** tandis que $(p \implies r) \land (q \implies r)$ est **F**.
\end{itemize}

Cette différence montre que les deux expressions ne sont pas logiquement équivalentes.


\vspace{0.5cm}

b) Soient les expressions 
$$A := (p \implies (q \land r)) \implies ((p \implies q) \land (p \implies r))$$ 
et 
$$B := (p \implies (q \land r)) \leftarrow ((p \implies q) \land (p \implies r))$$

Voici un tableau de vérité pour montrer que l'expression est vraie.

\begin{center}
\begin{tabular}{| c | c | c | c | c | c | c | c | c | c |}
\hline
$p$ & $q$ & $r$ & $p \implies (q \land r)$ & $q \land r$ & $A$ & $p \implies q$ & $(p \implies q) \land (p \implies r)$ & $p \implies r$ & $B$\\
\hline
V & V & V & V & V & V & V & V & V & V\\
V & V & F & F & F & F & V & F & F & F\\
V & F & V & F & F & F & F & F & V & F\\
F & V & V & V & V & V & V & V & V & V\\
V & F & F & F & F & F & F & F & F & F\\
F & F & V & V & F & V & V & V & V & V\\
F & V & F & V & F & V & V & V & V & V\\
F & F & F & V & F & V & V & V & V & V\\
\hline
\end{tabular}
\end{center}

\begin{center}
\begin{tabular}{| c | c |}
\hline
$A \land B$ & $(p \implies (q \land r)) \iff ((p \implies q) \land (p \implies r))$\\
\hline
V & V \\
F & F \\
F & F \\
V & V \\
F & F \\
V & V \\
V & V \\
V & V \\
\hline
\end{tabular}
\end{center}

par la définition de l’opérateur si et seulement si.

\end{document}
