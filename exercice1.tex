\documentclass{article}
\usepackage{amsmath}
\usepackage[margin=1in]{geometry}
\usepackage{array} 

\begin{document}

Exercice 1

\vspace{0.5cm}

a) Soit $p := 1 < 0, q := 2 < 3$ et $r := 2 < 1$.
Nous pouvons voir que l'expression 

$$(p \land q) \implies r$$
est vraie parce que $p \land q$ est fausse, étant une conjonction contenant une affirmation fausse, donc par définition, l'implication est vraie. D'autre part, l'expression 

$$ (p \implies r) \land (q \implies r)$$
est fausse parce que q est vraie mais r fausse et l'expression est une conjonction. Or, parce qu'elles diffèrent en valeurs de vérité pour les mêmes variables, ce ne sont pas des expressions équivalentes.

\vspace{0.5cm}
Il est également possible de le prouver en effecturant la table de vérité de l'expression:

\begin{center}
\begin{tabular}{| c | c | c | c | c | c | c | c | c |}
\hline
$p$ & $q$ & $r$ & $p \land q$ & ($p \land q$) \implies r & $ $p \implies $r $& $q \implies r  $ & $ (p \implies r) \land (q \implies r) $ \\
\hline
V & V & V & V & V & V & V & V\\
V & V & F & V & F & F & F & F\\
V & F & V & F & V & V & V & V\\
V & F & F & F & V & F & V & F\\
F & V & V & F & V & V & V & V\\
F & V & F & F & V & V & F & F\\
F & F & V & F & V & V & V & V\\
F & F & F & F & V & V & V & V\\
\hline
\end{tabular}
\end{center}

\vspace{0.5cm}

Il est possible de voir que pour certaines valeurs de p, q et r, l'expression est vraie.

\vspace{0.5cm}

b) Soient les expressions $$A:= (p \implies (q \land r)) \implies ((p \implies q) \land (p \implies r))$$ et
$$B := (p \implies (q \land r)) \impliedby ((p \implies q) \land (p \implies r))$$

Voici un tableau de vérité pour montrer que l'expression est vraie.

\begin{center}
\begin{tabular}{| c | c | c | c | c | c | c | c | c | c | c |}
\hline
$p$ & $q$ & $r$ & $p \implies$ ($q \land r$) & $q \land r$ & $ A $& $p \implies q  $ & $ (p \implies q) \land (p \implies r) $ & $p \implies r $ & $ B $\\
\hline
V & V & V & V & V & V & V & V & V & V\\
V & V & F & F & F & V & V & F & F & V\\
V & F & V & F & F & V & F & F & V & V\\
F & V & V & V & V & V & V & V & V & V\\
V & F & F & F & F & V & F & F & F & V\\
F & F & V & V & F & V & V & V & V & V\\
F & V & F & V & F & V & V & V & V & V\\
F & F & F & V & F & V & V & V & V & V\\
\hline
\end{tabular}
\end{center}

\begin{center}
\begin{tabular}{| c | c |}
\hline
$ A \land B$ & $(p \implies (q \land r)) \iff ((p \implies q) \land (p \implies r))$\\
\hline
V & V \\
V & V \\
V & V \\
V & V \\
V & V \\
V & V \\
V & V \\
V & V \\
\hline
\end{tabular}
\end{center}

par la définition de l’opérateur si et seulement si.



\end{document}
