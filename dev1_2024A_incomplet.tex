\documentclass[12pt]{article}

\newcommand{\ENONCE}[1]{#1}
 \newcommand{\REMISE}[1]{}
 \usepackage[T1]{fontenc}
\frenchspacing
\usepackage[francais]{babel}  % genere les trucs automatique en francais, dont les beaux guillemets
\usepackage[utf8]{inputenc}
\usepackage[top=25mm, bottom=25mm, left=25mm, right=25mm]{geometry}

\usepackage{amssymb}
\usepackage{amsmath}
\usepackage{epigraph}
%\usepackage{subfig}
\usepackage{framed}
\usepackage{graphicx}
\usepackage{color}
\usepackage{tikz}%%%%%%%pour les diagrammes de Venn mais franchement ne vous donnez pas ce trouble?

%%% Couleurs du texte %%%
\newcommand{\rouge}[1]{\textcolor{red}{#1}}
\newcommand{\bleu}[1]{\textcolor{blue}{#1}}
\newcommand{\blanc}[1]{\textcolor{white}{#1}}

%%% Valeurs booléennes %%%
\newcommand{\vrai}{\mbox{\tt vrai}}
\newcommand{\faux}{\mbox{\tt faux}}
\newcommand{\V}{\mbox{\tt v}}
\newcommand{\F}{\mbox{\tt f}}

%%% Opérateur booléens %%%
\newcommand{\non}{\neg}                  % Opérareur de négation
\newcommand{\et}{\wedge}                 % Opérareur de conjonction
\newcommand{\ou}{\vee}                   % Opérareur de disjonction
\newcommand{\ouexclusif}{\veebar}        % Opérareur de ou exclusif
\newcommand{\implique}{\Rightarrow}      % Opérareur d'implication
\newcommand{\impliqueinv}{\Leftarrow}    % Opérareur d'implication inverse
\newcommand{\ssi}{\Leftrightarrow}       % Opérareur si et seulement si

%%% Opérateurs ensemblistes %%%
\newcommand{\dans}{\in}                  % Opérateur d'appartenance
\newcommand{\inclus}{\subseteq}          % Opérareur d'inclusion
\newcommand{\strictinclus}{\subset}      % Opérareur d'inclusion stricte
\newcommand{\inclusinv}{\supseteq}       % Opérareur d'inclusion inverse
\newcommand{\strictinclusinv}{\supset}   % Opérareur d'inclusion stricte inverse

\newcommand{\nondans}{\notin}              % Négation de l'opérateur d'appartenance
\newcommand{\noninclus}{\not\subseteq}     % Opérareur d'inclusion
\newcommand{\nonstrictinclus}{\not\subset} % Opérareur d'inclusion stricte

\newcommand{\inter}{\cap}                 % Opérateur d'intersection
\newcommand{\union}{\cup}                 % Opérateur d'intersection
\newcommand{\comp}{{\mbox{\footnotesize $c$}}} % Complément
\newcommand{\moins}{\setminus}

%%% Ensembles fréquemments utilisés %%%
\newcommand{\ensembleVide}{\emptyset}  % L'ensemble vide
\newcommand{\ensembleU}{{\mathbf{U}}}  % L'ensemble universel
\newcommand{\ensembleB}{{\mathbb{B}}}  % L'ensemble des nombres booléens
\newcommand{\ensembleN}{{\mathbb{N}}}  % L'ensemble des nombres naturels
\newcommand{\ensembleZ}{{\mathbb{Z}}}  % L'ensemble des nombres relatifs
\newcommand{\ensembleR}{\mathbb{R}}    % L'ensemble des nombres reels
\newcommand{\ensembleQ}{\mathbb{Q}}    % L'ensemble des nombres rationnels

%%% Relations  %%%
\newcommand{\compose}{\circ}            % Opérateur de composition
\newcommand{\croix}{\times}             % Opérateur de produit cartésien
\newcommand{\Identite}{{\mathbf{I}}}    % Relation Identitée
\newcommand{\Domaine}{{\mbox{\rm Dom}}} % Domaine
\newcommand{\Image}{{\mbox{\rm Im}}}    % Image
\newcommand{\tuple}[1]{\ensuremath{\left\langle #1 \right\rangle}}  % n-tuple
\newcommand{\Rcal}{{\mathcal R}}       % «R» caligraphique
\newcommand{\Lcal}{{\mathcal L}}       % «L» caligraphique

%%% Démonstrations %%%
\newcommand{\cqfd}{\blanc{.}\\[-2mm]\mbox{}\hfill {\bf C.Q.F.D.}\\} % C.Q.F.D
\newcommand{\EQUIVALENT}[1]{ \quad \left\langle \mbox{ \it #1 } \right\rangle }
\newcommand{\EXPLICATION}[1]{ \blanc{.}\hfill $\left\langle \mbox{ \small\it #1 } \right\rangle$ }

%%% Divers %%%
\newcommand{\textdef}[1]{{\sl\bf #1}\index{#1}}         % Définition de terme
\newcommand{\eqdef}{\overset{{\mbox{\rm\tiny def}}}{=}} % Symbole de définition
\newcounter{exercice}\newcommand{\exercice}{\bigskip \addtocounter{exercice}{1}\noindent \textbf{Exercice \theexercice}\\}
\newcommand{\reponse}[1]{\REMISE{\vspace{.5cm}\noindent\textbf{Réponse : } #1 \vfill}}
\REMISE{\renewcommand{\ENONCE}[1]{}}
\newcommand{\solution}[1]{}

\begin{document}
% ---------------------------------- DÉBUT DES MODIFICATIONS À FAIRE ---------------------------
% Décommenter la ligne suivante pour compiler en mode ``Remise de travail''
%\renewcommand{\REMISE}[1]{#1} 

% ----------------------------------  CONSEILS POUR FACILITER LA REMISE  ---------------------------
% 
%   Les noms ne sont pas nécessaires (les fichiers sont liés à l'équipe dans le portal) Nous les avons insérés quand même car plusieurs étudiants sont plus à l'aise d'avoir une copie « normale »
%
%   Compilez votre devoir en un fichier et [utilisez un éditeur de pdf pour séparer vos fichiers en numéros
%   		 ou par exemple « imprimer en pdf p1,2 » pour le numéro 1, etc]
%
%   ---------------------------------------------------------------------

\REMISE{
\noindent
\begin{tabular}{l}
% Modifiez les noms etc  ---------------------------------------------------------------------
Nom1  B-GLO ou B-IFT\\%pas nécessaire
Nom2  B-GLO ou B-IFT
\end{tabular}
}
\hfill
 {\large MAT-1919:  Automne 2024}\\
\begin{center}
{\LARGE \textbf{DEVOIR 1}}\\[2mm]
\emph{À remettre \emph{ le 30 septembre 2024} à 23h59 au plus tard}
\end{center}
\REMISE{\newpage}% 

\ENONCE{
\noindent
\emph{
Toutes les consignes suivantes seront considérées dans la note. }
\begin{itemize}
\item[$\bullet$]  Créez votre équipe (de 1 à 3 étudiants) avant la date limite de création d'une équipe, \textbf{dans tous les cas}, même si vous êtes seul et même si \textcolor{lightgray}{ \it \tt [insert reason here]}.%, même si le ciel est bleu. 
\item[$\bullet$]  Remise :  {\large\rouge{\textbf{un} fichier pdf \textbf{par exercice} (un seul)\footnote{En fait vous rencontrerez un problème si on s'emballe pour le nombre de numéro : le monportail ne vous permet pas de remettre plus de 5 fichiers à la fois. Qu'à cela ne tienne! vous le ferez en 2 coups!}}, chacun   identifié par son numéro comme \bleu{dernier} caractère} (ex.: \emph{1.pdf, no2.pdf, mat1919dev2no5.pdf}). Nous utilisons un programme pour gérer les fichiers. Si plus d'un fichier termine par 3, ça crée problème.
\item[$\bullet$]  Soignez la \textbf{lisibilité} et l'orthographe. Les photos sont souvent  de \textbf{piètre qualité}. Des logiciels de numérisation pour téléphone font mieux, comme OfficeLens.
\item[$\bullet$]  Retard :  les 2 premières heures non pénalisées; ensuite -1\% par heure de retard.
\end{itemize}
\noindent
Il n'est pas nécessaire de remettre une page couverture, ni de mettre vos noms ou matricules. Les noms sont fixés aux équipes et tout ça est associé automatiquement par la remise sur le portail. 
}



%%%%%%%%%%%%%%%%%%%%%%%%%%%%%%%%%%%%%%%%%%%%%%%%%%%%%%%%%%
\exercice
a) Démontrez que l'implication n'est pas distributive à droite sur la conjonction. C'est-à-dire démontrez
$$(p\wedge q)\implique r \mbox{ n'est pas équivalent à } (p\implique r ) \wedge (q\implique r).$$
\\b) Démontrez que pourtant,  l'implication est  distributive à gauche sur la conjonction! C'est-à-dire démontrez par table de vérité qu'on a bien
$$p\implique (q\wedge r)\  \ssi\  (p\implique q ) \wedge (p\implique r).$$
\reponse{
}

%%%%%%%%%%%%%%%%%%%%%%%%%%%%%%%%%%%%%%%%%%%%%%%%%%%%%%%%%%

% % % % EXEMPLE DE TABLE DE VERITE % % % %
%\begin{center}
%\begin{tabular}{c c c c|c|c|c|c|c|c}
%$x_1$ & $x_2$ & $x_3$ & $x_4$ & $\overbrace{x_1\ou x_2}^{c_1}$ & $\overbrace{\non x_1 \ou x_3}^{c_2}$ & $\overbrace{x_1 \ou \non x_2}^{c_3}$ & $\overbrace{\non x_2 \ou x_3 }^{c_4}$ & $\overbrace{\non x_1 \ou \non x_3 }^{c_5}$ & $\overbrace{c_1 \et c_2 \et c_3 \et c_4 \et c_5}^{\psi_c}$ \\
%\hline
%\V & \V & \V & \V & \V & \V & \V & \V &\F & \rouge{\F} \\
%\V & \V & \V & \F & \V & \F & \V & \F &\V & \rouge{\F} \\
%\V & \V & \F & \V & \V & \V & \V & \V &\F & \rouge{\F} \\
%\V & \V & \F & \F & \V & \F & \V & \V &\V & \rouge{\F} \\
%\V & \F & \V & \V & \V & \V & \F & \V &\V & \rouge{\F} \\
%\V & \F & \V & \F & \V & \V & \F & \F &\V & \rouge{\F} \\
%\V & \F & \F & \V & \F & \V & \V & \V &\V & \rouge{\F} \\
%\V & \F & \F & \F & \F & \V & \V & \V &\V & \rouge{\F} \\
%\F & \V & \V & \V & \V & \V & \V & \V &\F & \rouge{\F} \\
%\F & \V & \V & \F & \V & \F & \V & \F &\V & \rouge{\F} \\
%\F & \V & \F & \V & \V & \V & \V & \V &\F & \rouge{\F} \\
%\F & \V & \F & \F & \V & \F & \V & \V &\V & \rouge{\F} \\
%\F & \F & \V & \V & \V & \V & \F & \V &\V & \rouge{\F} \\
%\F & \F & \V & \F & \V & \V & \F & \F &\V & \rouge{\F} \\
%\F & \F & \F & \V & \F & \V & \V & \V &\V & \rouge{\F} \\
%\F & \F & \F & \F & \F & \V & \V & \V &\V & \rouge{\F}
%\end{tabular}
%\end{center}
%
\exercice
a) L'expression $\non (p \ou q\implique  r) $ est-elle satisfiable? est-elle toujours vraie?  est-elle toujours fausse? Justifiez   en donnant soit des valuations qui témoignent de vos affirmations, soit par table de vérité, selon ce qui est le plus approprié. Faites des phrases complètes!
\\
b) Transformez cette expression en FNC forme normale conjonctive (voir CNF p.29 dans les notes de cours) en incluant la démonstration. En transformant il se peut que vous trouviez des façons de simplifier l'expression, ce qui n'est pas nécessaire. Vous pouvez arrêter dès que l'expression est en FNC;  \textbf{toutefois, si vous continuez à simplifier}, indiquez toutes les lignes qui contiennent des expressions en FNC.
\reponse{
}
\vfill




%-------------------------- Exemple démonstration par succession d'équivalences ---------------
%\paragraph{Démonstration de la proposition 1.1.8} (Contraposition) \\
%\noindent
%Soit $p$ et $q$ deux expressions booléennes.  Démontrons ``$p \implique q \ \ssi \ \non q \implique \non p$'':\\[-8pt]
%\begin{equation*}
%\begin{array}{ll}
% \hspace{.5cm}& p \implique q  \\
%%
% \ssi & \EQUIVALENT{Déf 1.1.2-a -- Définition de l'implication} \\
% & \non p \ou q \\
%%
% \ssi & \EQUIVALENT{Prop 1.1.5-c -- Double négation, avec $[p:=q]$}\\
% & \non p \ou \non(\non q) \\
%%
% \ssi & \EQUIVALENT{Prop 1.1.7-d -- Commutativité de la disjonction, avec $[p:=\non p]$ et $[q:=\non(\non q)]$} \\
% & \non(\non q) \ou  \non p \\
%%
% \ssi & \EQUIVALENT{Déf 1.1.2-a -- Définition de l'implication, avec $[p:=\non q]$ et $[q:=\non p]$} \\
% & \non q \implique \non p
%\end{array}
%\end{equation*}
%\cqfd
%

\newpage
\exercice
\emph{Notez que pour ce numéro, pour alléger la correction, seulement certains numéros, choisis au hasard seront lus, notés et commentés. Les mêmes pour tous les étudiants.}

Nous voulons développer un nouveau jeu de cartes. Nous désirons modéliser les cartes comme une base de données, mais ça, c'est pour le devoir 2; pour le moment, voyons une partie de ce que nous y manipulerons.
%L'avantage de définir les commandes suivantes dans un mbox en laTeX ici, c'est que même dans un environnement math les lettres vont avoir la même allure qu'hors de l'environnement. Dans un texte bien écrit, on ne veut pas qu'une variable/constante mathématique change d'apparence. De plus, dans un environnement mathématique, l'espace entre les lettres est calculé comme si les lettres étaient des variables multipliées et non comme un mot. Ce n'est donc pas acceptable pour les mots.
\newcommand{\role}{\mbox{ROLE}}
\newcommand{\attribut}{\mbox{ATTRIBUT}}
\newcommand{\attributn}{\mbox{ATTRIBUT\_A}}
\newcommand{\profil}{\mbox{PROFIL}}

Considérons les ensembles suivants:
\begin{itemize}
\item L'ensemble \role\ contenant les rôles (on pourrait penser à \textsf{humain},  \textsf{gnome}, \textsf{lion}, \textsf{oiseau}).
\item L'ensemble \attribut\ contenant les différents attributs (par exemple \textsf{deux\_mains},  \textsf{museau}, \textsf{bec}, \textsf{plume}).
\item L'ensemble \attributn\ contenant des attributs alimentaires (qui servent à s'alimenter), parmi lesquels \textsf{bec}.
\item L'ensemble \profil\ contenant des regroupements d'attributs, c'est-à-dire \profil$\subseteq \mathcal{P}(\attribut)$.
\end{itemize}
\noindent
Les règles du jeu n'ont pas d'importance pour le moment. 

Pour chaque phrase suivante, écrivez une expression qui la représente. Pour un défi supplémentaire (non pénalisé) utilisez seulement les opérateurs ensemblistes, si c’est possible. 
\\[2mm]
a) \textsf{bec} est un attribut.\\[2mm]
b) Les attributs alimentaires sont des attributs.\\[2mm]
c) Les attributs ne sont pas tous alimentaires.\\[2mm]
d) L'ensemble des profils qui contiennent  la \textsf{plume}\\[2mm]
 e) Il y a au moins 20 attributs qui ne sont pas alimentaires.\\[2mm]
 f) L'ensemble des attributs qui ne font partie d'aucun profil.\\[2mm]
 g)  Tout attribut fait partie d'au moins un profil.\\[2mm]
h) Aucun rôle n'est un attribut, et aucun attribut n'est un rôle.
 \\Conseil : imaginez-en d'autres, car il y aura une question semblable à l'examen! \\[3mm]
 Répondez aussi aux questions suivantes\\[2mm]
i) 
Quelle interprétation possède l'expression \\\mbox{}\hfill$(\forall x \dans \attributn \mid (\exists b\in \profil\mid  \neg(x\in b) \,) )\,.$\hfill\mbox{}
 \\[3mm]
j) Mettez le bon symbole $\in,\subseteq, \supseteq$  entre les éléments suivants. Si rien n’est possible, indiquez-le. \\[-3mm]
\begin{itemize}
\item[f-1.] $\textsf{plume} ~~???~~ \mathcal{P}(\attribut)$
\item[f-2.] $\attributn~~???~~ \mathcal{P}(\attribut)$
\item[f-3.] $\{\textsf{bec}\}~~???~~\profil$
\item[f-4.] $\mathcal{P}(\attributn) ~~???~~ \mathcal{P}(\attribut)$
\end{itemize}
Si un symbole est possible \emph{syntaxiquement} mais qu'il manque d'information pour dire si c'est vrai ou faux,   mettez-en un quand même. Par exemple, on ne sait pas si dans les profils il y a une profil vide, mais comme c'est possible, on pourrait  écrire $\emptyset\in\profil$. Ainsi vous n'avez pas non plus à tenir compte des affirmations a) à i).





%%-------------------------- Exemple d'écritures de diagramme de Venn  ---------------
%   Mais franchement slm si ça vous amuse! un beau dessin numérisé est tout aussi bien
% vous pouvez aussi faire dans un autre logiciel et insérer l'image comme ceci.
%\mbox{}\hfill\includegraphics[scale=0.3]{votreIMAGE}\hfill\mbox{}

%% Definition of circles
%\def\scircle{(0,0) circle (0.75cm)}
%\def\tcircle{(0:1cm) circle (0.75cm)}
%\def\vcircle{(2cm:1cm) circle (0.75cm)}
%
%\colorlet{circle edge}{blue!50}
%\colorlet{circle area}{blue!20}
%
%\tikzset{filled/.style={fill=circle area, draw=circle edge, thick},
%    outline/.style={draw=circle edge, thick}}
%
%%\setlength{\parskip}{5mm}
%\begin{center}
%\begin{minipage}{.2\textwidth}
%\begin{tikzpicture}
%    \begin{scope}
%        \fill[filled] \scircle;
%    \end{scope}
%    \draw[outline] \scircle node {$S$};
%    \draw[outline] \tcircle node {$T$};
%    \draw[outline] \vcircle node {$V$};
%    
%    \draw  ([xshift=-15pt,yshift=15pt]current bounding box.north west) 
%      rectangle ([xshift=15pt,yshift=-15pt]current bounding box.south east);
%
%	\node[xshift=10pt,yshift=10pt] at (current bounding box.south west) {$\mathbf{U}$};
%    \node[anchor=south] at (current bounding box.north) {$S$};
%    
%\end{tikzpicture}
%\end{minipage}%
%\begin{minipage}{.1\textwidth}
%$\quad\quad \cap$
%\end{minipage}%
%\begin{minipage}{.2\textwidth}
%$\left(\begin{array}{c}
%\begin{tikzpicture}
%    \begin{scope}
%        \fill[filled] \tcircle;
%        \fill[filled] \vcircle;
%    \end{scope}
%    \draw[outline] \scircle node {$S$};
%    \draw[outline] \tcircle node {$T$};
%    \draw[outline] \vcircle node {$V$};
%    
%    \draw  ([xshift=-15pt,yshift=15pt]current bounding box.north west) 
%      rectangle ([xshift=15pt,yshift=-15pt]current bounding box.south east);
%
%	\node[xshift=10pt,yshift=10pt] at (current bounding box.south west) {$\mathbf{U}$};
%    \node[anchor=south] at (current bounding box.north) {$T \cup V$};
%    
%\end{tikzpicture}\end{array}\right)$
%\end{minipage}%
%\begin{minipage}{.15\textwidth}
%$\qquad\qquad =$
%\end{minipage}%
%\begin{minipage}{.3\textwidth}
%\begin{tikzpicture}\quad
%    \begin{scope}
%    	\clip \scircle;
%        \fill[filled] \tcircle;
%        \fill[filled] \vcircle;
%    \end{scope}
%    \draw[outline] \scircle node {$S$};
%    \draw[outline] \tcircle node {$T$};
%    \draw[outline] \vcircle node {$V$};
%    
%    \draw  ([xshift=-15pt,yshift=15pt]current bounding box.north west) 
%      rectangle ([xshift=15pt,yshift=-15pt]current bounding box.south east);
%
%	\node[xshift=10pt,yshift=10pt] at (current bounding box.south west) {$\mathbf{U}$};
%    \node[anchor=south] at (current bounding box.north) {$S \cap (T \cup V)$};
%    
%\end{tikzpicture}
%\end{minipage}%
%\end{center}

\textcolor{red}{Autres numéros à venir!}
\end{document}
